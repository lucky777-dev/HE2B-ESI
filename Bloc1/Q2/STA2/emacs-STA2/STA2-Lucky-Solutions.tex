% Created 2021-03-10 mer 01:59
% Intended LaTeX compiler: pdflatex
\documentclass[french]{article}
\usepackage{minted}
\usepackage[round]{natbib}
\usepackage{babel}
\usepackage[utf8]{inputenc}
\usepackage[T1]{fontenc}
\usepackage{mathptmx}
\usepackage{xfrac}
\usepackage{booktabs}
\usepackage{longtable}
\usepackage[skip=1.\baselineskip]{caption}
\usepackage{soul}
\usepackage[usenames,dvipsnames,svgnames]{xcolor}
\usepackage{parskip}
\usepackage[many]{tcolorbox}
\usepackage{hyperref}
\usepackage[export]{adjustbox}
\usepackage{subcaption}
\definecolor{luckydarkpurple}{RGB}{142,0,142}
\definecolor{luckydarkgreen}{RGB}{0,142,0}
\definecolor{luckydarkblue}{RGB}{0,0,142}
\hypersetup{
    colorlinks=true,
    linkcolor={luckydarkpurple},
    citecolor={luckydarkgreen},
    urlcolor={luckydarkblue}}
\usepackage{geometry}
\geometry{a4paper,left=2.5cm,top=1.2cm,right=2.5cm,bottom=1.5cm,marginparsep=7pt, marginparwidth=.6in}
\usepackage[utf8]{inputenc}
\usepackage[T1]{fontenc}
\usepackage{graphicx}
\usepackage{grffile}
\usepackage{longtable}
\usepackage{wrapfig}
\usepackage{rotating}
\usepackage[normalem]{ulem}
\usepackage{amsmath}
\usepackage{textcomp}
\usepackage{amssymb}
\usepackage{capt-of}
\usepackage{hyperref}
\author{Sm!le42}
\date{\today}
\title{STA2 - Lucky Solutions}
\hypersetup{
 pdfauthor={Sm!le42},
 pdftitle={STA2 - Lucky Solutions},
 pdfkeywords={},
 pdfsubject={},
 pdfcreator={Emacs 26.3 (Org mode 9.1.9)}, 
 pdflang={Frenchb}}
\begin{document}

\maketitle
\tableofcontents


\section{Chapitre 1 -- Introduction au calcul des probabilités}
\label{sec:orgc77019c}

\subsection{Exercice 3}
\label{sec:orgce2b2e1}

Un dé est truqué de sorte qu’en le lançant, la probabilité d’obtenir 6 vaut le triple de celle d’obtenir toute autre valeur. Avec ce dé, quelle est la probabilité d’obtenir un point pair ?

\subsubsection{Solution \emph{(\(\sfrac{5}{8}\))}}
\label{sec:orgc760d17}

\begin{center}
\begin{tabular}{llllll}
P(1) = \(\sfrac{1}{8}\) & P(2) = \(\sfrac{1}{8}\) & P(3) = \(\sfrac{1}{8}\) & P(4) = \(\sfrac{1}{8}\) & P(5) = \(\sfrac{1}{8}\) & \textbf{P(6) =} \(\mathbf{\sfrac{3}{8}}\)\\
\end{tabular}
\end{center}

\begin{center}
\textbf{P(Pair)} = P(2)+P(4)+P(6)

= \(\sfrac{1}{8}\) + \(\sfrac{1}{8}\) + \(\sfrac{3}{8}\)

\(\textcolor{luckydarkgreen}{=\mathbf{\sfrac{5}{8}}}\)
\end{center}

\subsection{Exercice 4}
\label{sec:org27c804a}
Trois chevaux sont en course. Le premier à 2 fois plus de chances de gagner que le deuxième, celui-ci a aussi 2 fois plus de chances de gagner que le troisième. Quelles sont les probabilités de gagner de chacun des trois chevaux ?

\subsubsection{Solution \emph{(\(\sfrac{4}{7}\);\(\sfrac{2}{7}\);\(\sfrac{1}{7}\))}}
\label{sec:org5bcfe7a}

\begin{center}
\begin{tabular}{|c|c|c|c|}
\hline
Cheval & 1 & 2 & 3\\
\hline
Proba & 4*x & 2*x & x\\
\hline
\end{tabular}
\end{center}
\begin{center}
\emph{x} = \(\frac{1}{4+2+1}\) = \(\sfrac{1}{7}\)
\end{center}
\begin{center}
\begin{tabular}{|c|c|c|c|}
\hline
Cheval & 1 & 2 & 3\\
\hline
Proba & \(\textcolor{luckydarkgreen}{\bold{4/7}}\) & \(\textcolor{luckydarkgreen}{\bold{2/7}}\) & \(\textcolor{luckydarkgreen}{\bold{1/7}}\)\\
\hline
\end{tabular}
\end{center}

\subsection{Exercice 5}
\label{sec:org0d3aeeb}

Soit un jeu de 52 cartes dont on tire une carte au hasard. On définit les évènements aléatoires suivants:
\begin{itemize}
\item A : obtenir un as
\item B : obtenir une carte rouge
\item C : obtenir un cœur.
\end{itemize}

Définissez les évènements suivants et calculez-en la probabilité :
\begin{enumerate}
\item A \(\cap\) B
\item B \(\cap\) C
\item A \(\cup\) C
\item B \(\cup\) C
\end{enumerate}

\subsubsection{Solution \emph{(\(\sfrac{1}{26}\);\(\sfrac{1}{4}\);\(\sfrac{4}{13}\);\(\sfrac{1}{2}\))}}
\label{sec:orgc34b655}

\begin{itemize}
\item P(A) = \(\sfrac{4}{52}\) = \(\sfrac{1}{13}\)
\item P(B) = \(\sfrac{26}{52}\) = \(\sfrac{1}{2}\)
\item P(C) = \(\sfrac{13}{52}\) = \(\sfrac{1}{4}\)
\end{itemize}

\begin{enumerate}
\item \textbf{A \(\cap\) B} \emph{\(\textcolor{gray}{\text{--> As et Rouge}}\)}
\label{sec:orga84da8f}

= P(A \textbf{et} B) = \(\sfrac{1}{13}\)*\(\sfrac{1}{2}\)
\(\textcolor{luckydarkgreen}{\mathbf{= \sfrac{1}{26}}}\)

\item \textbf{B \(\cap\) C} \emph{\(\textcolor{gray}{\text{--> Rouge et Coeur}}\)}
\label{sec:org054730d}

= P(B \textbf{et} C) = P(C) \(\textcolor{luckydarkgreen}{\mathbf{= \sfrac{1}{4}}}\) \emph{\(\textcolor{gray}{\text{--> Car un coeur est toujours rouge}}\)}

\item \textbf{A \(\cup\) C} \emph{\(\textcolor{gray}{\text{--> As ou Coeur}}\)}
\label{sec:org4113b04}

= P(A \textbf{ou} C) = P(A)+P(C)--P(A \textbf{et} C) \emph{\(\textcolor{gray}{\text{--> On retire les As Rouges comptés en double}}\)}

= \(\sfrac{4}{52}\) + \(\sfrac{13}{52}\) -- \(\sfrac{4}{52}\)*\(\sfrac{1}{4}\)

\(\textcolor{luckydarkgreen}{\mathbf{= \sfrac{4}{13}}}\)

\item \textbf{B \(\cup\) C} \(\textcolor{gray}{\text{--> Rouge ou Coeur}}\)
\label{sec:org763e717}

= P(B \textbf{ou} C) = P(B) \emph{\(\textcolor{gray}{\text{--> Car un coeur est toujours rouge}}\)}

\(\textcolor{luckydarkgreen}{\mathbf{= \sfrac{1}{2}}}\)
\end{enumerate}

\subsection{Exercice 9}
\label{sec:org8efd4f7}

Soit un groupe composé de 12 hommes dont la moitié a des lunettes et de 15 femmes dont le tiers a des lunettes. Si on choisit une personne au hasard dans ce groupe, quelle est la probabilité que cette personne soit un homme ou porte des lunettes ?

\subsubsection{Solution \emph{(\(\sfrac{17}{27}\))}}
\label{sec:org07f250d}

\begin{center}
\begin{tabular}{c|c|c|c|}
 & Lunettes & !Lunettes & Total\\
\hline
Hommes & \(\textcolor{teal}{\bold{6}}\) & 6 & \(\textcolor{teal}{\bold{12}}\)\\
\hline
Femmes & 5 & 10 & 15\\
\hline
Total & \(\textcolor{teal}{\bold{11}}\) & 16 & \(\textcolor{teal}{\bold{27}}\)\\
\end{tabular}
\end{center}

--> P(Hommes \textbf{ou} Lunettes)

= P(Homme) + P(Lunettes) -- P(Homme \textbf{et} Lunettes)

= \(\sfrac{12}{27}\) + \(\sfrac{11}{27}\) -- \(\sfrac{6}{27}\)

\(\textcolor{luckydarkgreen}{\mathbf{= \sfrac{17}{27}}}\)

\begin{quote}
\textbf{Même résultat avec la Loi \emph{complémentaire} de Morgan:}

P(Homme ou Lunettes) = 1 -- P(Femmes et !Lunettes)

= 1 -- \(\sfrac{10}{27}\) \(\textcolor{luckydarkgreen}{\mathbf{= \sfrac{17}{27}}}\)
\end{quote}

\subsection{Exercice 10}
\label{sec:org2853394}

Lors de vacances scolaires, deux activités sportives sont proposées : natation et vélo. On sait que 40\% des participants se sont inscrits à la natation, 50\% aux randonnées vélo et 25\% se sont inscrits au deux. Quelle est la probabilité qu’un participant choisi au hasard ne fasse pas de sport ?

\subsubsection{Solution \emph{(35\%)}}
\label{sec:orgb97e475}

\begin{center}
\begin{tabular}{|c|c|c|c|}
\hline
 & Vélo & !Vélo & Total\\
\hline
Natation & 25\% & 15\% & 40\%\\
\hline
!Natation & 25\% & 35\% & 60\%\\
\hline
Total & 50\% & 50\% & 100\%\\
\hline
\end{tabular}
\end{center}

P(!Sport) = P(!Vélo \textbf{et} !Natation) \(\textcolor{luckydarkgreen}{\mathbf{= \sfrac{35}{100}}}\)

\begin{quote}
\textbf{Alternative:}

P(!Sport) = 1 -- P(Natation \textbf{ou} Vélo)

= 1 -- P(Natation) + P(Vélo) -- P(Natation \textbf{et} Vélo)

= 100\%--40\%+50\%--25\% \(\textcolor{luckydarkgreen}{\mathbf{= \sfrac{35}{100}}}\)
\end{quote}

\subsection{À VENIR \ldots{}}
\label{sec:org466b8e9}
\end{document}
